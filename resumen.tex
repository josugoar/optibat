\newpage

\thispagestyle{empty}

\begin{center}
  {\bf \huge Resumen}
\end{center}

\vspace{1cm}

En respuesta a la creciente adopción de los sistemas de almacenamiento de energía en baterías y la ausencia de soluciones actualmente en producción principalmente centradas en configuraciones topológicas híbridas, se define el diseño, desarrollo, despliegue y validación de \textsc{Optibat}, un sistema integral para la automatización del arbitraje en el mercado eléctrico. El sistema busca maximizar la rentabilidad económica comprando y vendiendo energía en los mercados spot mediante el control del ciclado de las baterías. Para ello, se integran datos operacionales en tiempo real de los activos energéticos con la información del entorno del operador del mercado y operador del sistema, empleando un modelo de programación lineal de enteros mixtos para determinar la estrategia óptima de carga y descarga e interactuando directamente con los agentes de mercado y operadores de telecontrol. Desplegado con éxito en múltiples instalaciones a gran escala, \textsc{Optibat} gestiona docenas de megavatios hora diarios y genera millones de euros de ingresos anuales previstos. Los resultados demuestran que las topologías híbridas son significativamente más rentables que las aisladas, destacando el modelo por su capacidad para aumentar el aprovechamiento de la generación energética. De esta forma, el proyecto aporta una solución robusta y escalable que valida el caso de negocio de las baterías, permitiendo su integración eficaz como activos óptimos y fiables en el mercado eléctrico.

\vspace{1cm}

\begin{center}
  {\bf \large Palabras clave}
\end{center}

\vspace{0.5cm}

Internet de las cosas industrial, sistema de almacenamiento de energía en baterías, mercado eléctrico, optimización energética, telecomunicación

\newpage

\thispagestyle{empty}

\begin{center}
  {\bf \huge Abstract}
\end{center}

\vspace{1cm}

In response to the increasing adoption of battery energy storage systems and the absence of solutions currently in production primarily focused on hybrid topological configurations, the design, development, deployment, and validation of Optibat is proposed, a comprehensive system for automating electricity market arbitrage. The system aims to maximize economic profitability by buying and selling energy on spot markets through battery cycling. To achieve this, the system integrates real-time operational data from energy assets with information from the environments of the market and system operators. It uses a mixed-integer linear programming model to determine the optimal charging and discharging strategy and directly aimed at market agents and telecommunications operators. Successfully deployed in multiple large-scale installations, Optibat manages dozens of megawatt-hours daily and generates millions of euros in projected annual revenue. The results demonstrate that hybrid topologies are significantly more profitable than isolated ones, highlighting the model's ability to increase the utilization of energy generation. This way, the project provides a robust and scalable solution that validates the business case for batteries, enabling their effective integration as optimal and reliable assets in the electricity market.

\vspace{1cm}

\begin{center}
  {\bf \large Keywords}
\end{center}

\vspace{0.5cm}

Industrial internet of things, battery energy storage system, electricity market, energy optimization, telecommunications
