\cleardoublepage%

\chapter*{Conclusions and future work}%
\label{makereference10}
\addcontentsline{toc}{chapter}{Introduction}

\section*{Conclusions}

In conclusion, a cross-market battery optimization system has been successfully designed, developed, deployed, and validated across multiple installations on the iberian market. The system manages dozens of megawatt hours daily and generates millions of euros in expected annual revenue by cycling battery energy storage systems, which satisfies the proposed main objective.

In turn, from the perspective of the fulfillment of secondary objectives, the maximization of profitability has been sought through the optimization of market positions, together with the consideration of battery degradation, cycling them in a stable and controlled manner through the business logic of the modeling. Furthermore, regulatory requirements have been met, mainly defined by the system operator in the form of technical limits, thus avoiding possible financial penalties. The security of the operation has been guaranteed by deploying part of the integration with the physical infrastructure in an industrial demilitarized zone, preventing the exposure of energy asset information to unauthorized networks. Also, the design maintains a generalized architecture, being able to operate in multiple installations without any modification of the system's logic. Improved facilities have been provided for market agents and telecommunication operators to monitor the system, and automatic response to failures, such as the lack of availability, has also been taken into account. Finally, an economic feasibility analysis has been carried out, showing measurable results.

On the other hand, from an architectural point of view, after analyzing the topological configurations with which the system works, the new operational infrastructure has been unified and implemented for both data acquisition from the battery energy storage system and the control of their charging and discharging, with the rest of the existing network components. Specifically, industrial communication protocols have been used to exchange information with the energy assets and introduce the integration of the batteries to the plant information system, which acts as the historian that saves the states of the physical signals over time for later consultation. With this, the data acquisition policy has been configured using spontaneous and general interrogation mechanisms. This way, a parallel is drawn to the well known phases of an industrial internet of things project: sensing, processing, management and actuation.

In addition to this, information about the market environment has been obtained through the interfaces provided by the corresponding official institutions, the market operator and transport system operator. The information has been processed and transformed into the appropriate format for mass storage in a data warehouse, which had to be queried efficiently. For this, periodic processes timed according to market schedules have been used.

Furthermore, a structural modeling of the physical scenario of the installations has been carried out, with the purpose of determining the optimal solution to the energy cycling problem. Using an abstract modeling language, the physical behaviors of all types of installations have been defined, from the least complex, such as standalone ones, to the most complex, such as colocated. Precisely, the logic of all energy assets, both storage and generation, has been incorporated into the model.

Finally, the command and control of the system have been appropriately specified, in the form of setpoints for selecting the power profile of the batteries and their corresponding return to technical minimum, along with the quoting of offer prices for buy and sell positions, using a bidding algorithm per half charge cycle. In addition, to facilitate supervision for market agents and telecommunication operators, a dashboard has been developed through which the automatic behavior of the system can be monitored, being able to take manual control if necessary.

With this, energy arbitrage in the market has been enabled for the installations the system has been deployed at. If it were not for it, as the market operator had was no other alternative solution currently deployed on the iberian market, the installations to be controlled would have lost the arbitrage opportunities or would have to be commanded manually by telecommunication operators, an absolutely inadmissible situation due to the complexity of the whole operation.

It is noteworthy that both battery energy storage systems and, therefore, cross-market battery optimization systems, are still in their infancy, at least in Europe. Precisely, according to a study on the use of storage technologies~\cite{hu2022potential}, ``the main energy storage system in Europe is still pumped hydroelectricity''. This means that, seeing how the landscape of electrical technologies is increasingly focusing on these battery technologies, it is likely that more sophisticated control systems will come to market. Although, currently, installations with battery energy storage systems are few and far between compared to the rest, market operators seek to incorporate them in all renewable installations to improve energy utilization. Therefore, while the implemented system has been successfully deployed throughout the country, it may be necessary to take new considerations in this regard and take into account said alternatives when the daily cycled energy might equal or exceed the amount of other well established energy assets'. The non-confidential censored code is available on GitHub\footnote{\url{https://github.com/josugoar/optibat}}.

\section*{Future work}

Regarding future work, \textsc{Optibat} is limited to arbitrage in the spot markets: daily, intraday, and continuous. However, there are other more or less profitable markets~\cite{cnmc2024balance}, such as the manual frequency restoration reserve and automatic frequency restoration reserve markets that trade availability. It would provide interesting adding support for such markets and complete the arbitrage profile of the battery storage technologies, as they are also ideal for the operation in said markets due to their high commutation capabilities.

While integration with regulation markets is definitely the most notable improvement to be made, refining the modeling of battery degradation also provides benefits~\cite{shamarova2022review}. By incorporating even more granular signals and the specifications of the battery management system (making the system battery specific), it is possible to model even the imperceptible degradation of the state of charge at rest. Certainly, although not being a high priority aspect, due to the constant cycling of the batteries which, by nature, does not precisely seek to maintain a steady state, the increased fidelity of the modeled physical processes would provide a substantial contribution. Furthermore, an in depth analysis of the degradation suffered by the battery, beyond the consumption of the cycles for which it is rated by the manufacturer, brings clarity for possible profit optimizations obtained under more adverse conditions than the ones experienced.

In addition, the most realistic future work comes in the form of integrating and deploying the optimization system in the growing number of new hybrid installations having energy storage assets, which are being continuously incorporated into the grid. Precisely and as has been previously detailed, battery energy storage systems are in continuous growth and, in fact, during the development process new storage systems have even become available.

It should be noted that efforts have been made for the maintenance and enhancement of the relevant work from the corresponding market operator, with the aim of guaranteeing the continuation of its life cycle, in a way that its maintenance and deployment continue under a new team's supervision.
