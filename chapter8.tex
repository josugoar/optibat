\cleardoublepage

\chapter{Conclusiones y trabajo futuro}
\label{makereference8}

\section{Conclusiones}
\label{makereference8.1}

En conclusión, se ha diseñado, desarrollado, desplegado y validado satisfactoriamente un sistema de optimización de sistemas de almacenamiento de energía en baterías real a lo largo de múltiples instalaciones del país, el cual mueve docenas de megavatios hora al día ciclando las baterías y genera millones de euros de ingresos esperados al año, representado en la figura~\ref{fig:arquitecuta-sistema}.

\begin{figure}
\centering
\includegraphics[width=0.5\linewidth]{figures/arquitecuta-sistema.png}
\caption{Arquitectura global del sistema}
\label{fig:arquitecuta-sistema}
\end{figure}

Desde la perspectiva del cumplimiento de objetivos, se ha buscado de la maximización de la rentabilidad mediante la optimización de las posiciones del mercado, junto con la minimización del desgaste de las baterías ciclándolas de forma estable mediante el algoritmo de oferta y limitando los mismos ciclos. Además, se han cumplido con los requisitos regulatorios principalmente dados por el operador del sistema, en forma de límites técnicos. Se ha garantizado la seguridad de la operación desplegando parte de la integración de la infraestructura en una zona desmilitarizada. También, el diseño cumple con los objetivos de generalización, siendo capaz de operar en múltiples instalaciones sin modificación alguna en la lógica del sistema. Se han dado facilidades a los agentes de mercado y operadores de telecontrol para la monitorización del sistema y se ha tenido en cuenta la respuesta automática ante fallos, como la falta de disponibilidad. Finalmente, se ha efectuado un análisis de viabilidad económica donde se muestran resultados medibles.

Por otro lado, según la perspectiva de la arquitectura, se ha unificado e implantado la nueva infraestructura operacional para tanto la adquisición de datos de la batería como el control de la carga y descarga de la misma con el resto de componentes ya existentes de la red. Concretamente, se ha hecho uso de protocolos de comunicación industriales para comunicarse con los activos energéticos e introducir la integración de los sistemas de almacenamiento de energía en baterías con el sistema de control de planta, que realiza la labor de historiador guardando los estados de las señales físicas a lo largo del tiempo para su posterior consulta.

Se ha obtenido la información del entorno de mercado a través de las instituciones oficiales correspondientes, el operador del mercado y el operador del sistema. La información se ha procesado y transformado en el formato adecuado para su almacenamiento masivo en bases de datos, la cual ha tenido que ser consultada eficientemente.

Se ha realizado un trabajo de modelización estructural de la situación física de las instalaciones, con el propósito de determinar la solución optima al problema de optimización de energía de las instalaciones. Mediante un lenguaje de modelado abstracto, se han definido los comportamientos físicos de todos los tipos de instalaciones, desde las menos complejas como las aisladas, a las más complejas como las híbridas.

Por último, se ha definido apropiadamente el apartado de comando y control, en forma de consignas de selección de potencia y vuelta al mínimo técnico de las baterías y ofertas de mercado de las posiciones de compra y venta de las mismas. Además, para facilitar la supervisión del sistema a los agentes de mercado, se ha desarrollado un cuadro de mando a través del cual monitorizar el comportamiento automático del sistema y tomar el control manual si es necesario.

De esta forma, se ha habilitado el arbitraje en el mercado óptimo de las instalaciones controladas. De no ser por el sistema desarrollado, al no existir ninguna solución alternativa de fácil y rápida implantación en la actualidad, las instalaciones a controlar tendrían que perder las oportunidades de arbitraje correspondientes o ser controladas manualmente por los operarios de telecontrol, situación absolutamente inadmisible.

Cabe destacar que tanto los sistemas de almacenamiento de energía en baterías como, por lo tanto, los sistemas de optimización de baterías en el mercado eléctrico se encuentran todavía dando sus primeros pasos. Esto significa que, viendo como el panorama de las tecnologías eléctricas pone su foco cada vez más en estas tecnologías de almacenamiento, es probable que salgan al mercado sistemas más sofisticados de control. Aunque en la actualidad las instalaciones con sistemas de almacenamiento de energía en baterías sean pocas en comparación, las empresas tecnológicas buscan implantarlas en toda las instalaciones de generación para mejorar el rendimiento de la energía. Aunque el sistema implantado ha sido satisfactoriamente desplegado a lo largo de todo el país, quizás sea necesario tomar nuevas consideraciones al respecto y tener en cuenta dichas alternativas cuando la energía movida al día no sea solo docenas de megavatios hora, sino cientos.

\section{Trabajo futuro}
\label{makereference8.2}

En cuanto al trabajo futuro, Optibat se encuentra limitado al arbitraje en los llamados mercados spot: el mercado diario, intradiarios y continuo. Aún así, existen otros mercados más o menos rentables como los mercados de regulación secundaria y terciaria que negocian disponibilidades. Resultaría interesante añadir soporte a estos mercados para completar así el perfil de arbitraje de estas tecnologías energéticas, ya que, coincidentalmente, las baterías también son idóneas para estos mercados debido a sus altas capacidades de conmutación.

Si bien la integración con los mercados de regulación es la mejora más notable sin lugar a dudas, también resultaría beneficioso mejorar la modelización del desgaste de las baterías. Mediante la incorporación de señales aún más granulares y las especificaciones del sistema de control de las baterías, sería posible modelizar incluso el imperceptible desgaste mismo del estado de carga de las baterías en reposo. Ciertamente, aunque no sea un tema sumamente prioritario debido al constante ciclado de las baterías que por naturaleza no busca precisamente que la batería esté quieta, la mejora sustancial vendría dada en forma de la mejora de fidelidad de los procesos físicos modelados. Además, un análisis en profundidad del desgaste sufrido por la batería, más allá del consumo de los ciclos para los que está calificada por el manufacturador, brindaría claridad a posibles mejoras del beneficio obtenido bajo condiciones más adversas que las experimentadas.

Junto a ello, el trabajo futuro más realista podría venir dado en forma de la implantación del sistema de optimización desarrollado en el creciente número de nuevas instalaciones con componentes de almacenamiento en incorporación continua a la red. Como ya se ha detallado anteriormente, los sistemas de almacenamiento de energía en baterías, al ser todavía una tecnología no tan ampliamente establecida, se encuentran en continuo crecimiento. De hecho, incluso durante el proceso de desarrollo se hizo disponible un nuevo sistema de almacenamiento.

Cabe destacar que se han realizado planes para continuar con el esfuerzo del trabajo desarrollado dentro de la entidad energética pertinente, con el objetivo de garantizar la continuación de su ciclo de vida, de tal forma que su operación y despliegue prosigan bajo la supervisión de un equipo diferente.
